\documentclass[a4paper]{article}
\usepackage{vntex}
\usepackage{a4wide,amssymb,epsfig,latexsym,array,hhline,fancyhdr}
\usepackage[normalem]{ulem}
\usepackage[makeroom]{cancel}
\usepackage{alltt}
\usepackage[framemethod=tikz]{mdframed}
\usepackage{caption,subcaption}
\usepackage{geometry}
\usetikzlibrary{arrows,snakes,backgrounds}
\usepackage[unicode]{hyperref}
\usepackage[normalem]{ulem}

\def\thesislayout{	
	\geometry{
		a4paper,
		total={170mm,240mm},
		left=20mm,
		top=30mm,
	}
}
\thesislayout
\begin{document}
	
	cắt từ các stock lớn nhất đến stock nhỏ nhất, từ product lớn nhất đến nhỏ nhất sao cho diện tích bị cắt của stock là thay đổi ít nhất.
	\begin{mdframed}[backgroundcolor=black!10]
		for $stock_i$ from the largest to smallest area in stock list:\\
		\hspace*{0.5cm}for $prod_j$ from the largest to smallest area in product list:\\
		\hspace*{1cm} for $(x,y)$ in all position can cut $prod_j$ from $stock_i$:\\
		\hspace*{1.5cm} let $S_{ij}$ be the area of the smallest rectangle contain cut area if cut $prod_j$ at position $(x,y)$:\\
		\hspace*{1cm} let $(x_{0},y_{0})$ be the position where the value of $S_{ij}$ is smallest\\
		\hspace*{1cm} cut the $prod_j$ from $stock_i$ at position $(x_{0},y_{0})$\\
		for $stock_i$ from the smallest to largest area in cut stock list:\\
		\hspace*{0.5cm}for $stock_j$ from the smallest to largest area in uncut stock list:\\
		\hspace*{1cm} if cut area of $stock_i$ is smaller than $stock_j$:\\
		\hspace*{1.5cm} move cut set of $stock_i$ to $stock_j$\\
	
\end{document}

